%Plantilla basada en "Template for Masters / Doctoral Thesis" (plantilla disponible en writeLaTex) que subió LaTeXTemplates.com

\documentclass[12pt]{report}
\usepackage[a4paper, top=2.5cm, bottom=2.5cm, left=2.5cm, right=2.5cm]{geometry}
\usepackage{amssymb,amsmath,amsthm} %paquete para símbolo matemáticos
\usepackage[spanish]{babel}
\usepackage[utf8]{inputenc} %Paquete para escribir acentos y otros símbolos directamente
\usepackage{enumerate}
\usepackage{graphicx}
%\usepackage{subfig} %para poner subfiguras
\graphicspath{{Img/}} %En qué carpeta están las imágenes
\usepackage[nottoc]{tocbibind}
\usepackage[pdftex,
            pdfauthor={Herbert Arias},
            pdftitle={Plantilla},
            pdfsubject={ÁREA DE LA TESIS},
            pdfkeywords={PALABRAS CLAVE},
            pdfproducer={Latex con hyperref},
            pdfcreator={pdflatex},
            hidelinks]{hyperref}

\renewcommand{\familydefault}{\sfdefault}
\usepackage[scaled=0.96]{helvet}
\usepackage[helvet]{sfmath}
\everymath={\sf}

\usepackage{lipsum} 
\usepackage{apacite} 


\begin{document}

%----------------------------------------------------------------------------------------
%	COMANDOS PERSONALIZADOS
%----------------------------------------------------------------------------------------

%SI TU TESIS TIENE TEOREMAS Y DEMOSTRACIONES, PUEDES DESCOMENTAR Y USAR LOS SIGUIENTES COMANDOS

%\renewcommand{\proofname}{Demostración}
%\providecommand{\norm}[1]{\lVert#1\rVert} %Provee el comando para producir una norma.
%\providecommand{\innp}[1]{\langle#1\rangle} 
%\newcommand{\seno}{\mathrm{sen}}
%\newcommand{\diff}{\mathrm{d}}

%\newtheorem{teo}{Teorema}[section] 
%\newtheorem{cor}[teo]{Corolario}
%\newtheorem{lem}[teo]{Lema}

%\theoremstyle{definition}
%\newtheorem{dfn}[teo]{Definición}

%\theoremstyle{remark}
%\newtheorem{obs}[teo]{Observación}

%\allowdisplaybreaks


%----------------------------------------------------------------------------------------
%	PORTADA
%----------------------------------------------------------------------------------------

\title{\Huge SOFTWARE EN EDUCACIÓN PRIMARIA} %Con este nombre se guardará el proyecto en writeLaTex

\begin{titlepage}
   \begin{center}

      {\textbf{\fontsize{0.65cm}{0.65cm}\selectfont UNIVERSIDAD NACIONAL MAYOR DE SAN MARCOS}}\\
      {\large (Universidad del Perú, DECANA DE AMÉRICA)}\\
      \textbf{\Large FACULTAD DE INGENIERÍA DE SISTEMAS E INFORMÁTICA}\\[0.7em]
      \textbf{{\large E.A.P. INGENIERÍA DE SOFTWARE}}

      %Figura
      \begin{figure}[h]
         \begin{center}
            \includegraphics[width=5.5cm]{san_marcos.png}
         \end{center}
      \end{figure}

      \textbf{\Large CURSO}\\[0.7em]
      {\Large REALIDAD NACIONAL Y MUNDIAL}\\
      \vspace{2em}
      \textbf{\Large MONOGRAFÍA}\\[0.7em]
      {\Large SOFTWARE EN EDUCACIÓN PRIMARIA}\\
      \vspace{2em}
      \textbf{\Large PROFESOR}\\[0.7em]
      {\Large Chaupi Torres, José Antonio}\\
      \vspace{2em}
      \textbf{\Large INTEGRANTES}\\[0.7em]
      {\Large Tarmeño Noriega Carlos Daniel}\\[0.5em]
      {\Large Tirado Julca Juan José}\\[0.5em]
      {\Large Ñontol Lozano, Rafael Santiago}\\[0.5em]
      {\Large Cortés Rosas, Ingrid Fiorella}\\[0.5em]
      {\Large Arias Silva Herbert Brice}\\
      \vspace{2em}
      {\Large Lima - Peru}\\[0.7em]
      \textbf{\Large 2018}
   \end{center}
\end{titlepage}

% ------------------------------------------------------------------------------
%  TABLA DE CONTENIDOS
% ------------------------------------------------------------------------------

\tableofcontents


% ------------------------------------------------------------------------------
%  TESIS
% ------------------------------------------------------------------------------

%  Introduccion
%  -----------------------------------------------------------------------------
\include{./Capitulos/Intro}
\thispagestyle{empty}

%  Capítulo 1
%  -----------------------------------------------------------------------------

%%=====================================================================================
%%
%%       Filename:  Intro.tex
%%
%%    Description:  Introducción de la Tesis
%%
%%        Version:  1.0
%%        Created:  11/22/2018
%%       Revision:  none
%%
%%         Author:  YOUR NAME (), 
%%   Organization:  
%%      Copyright:  Copyright (c) 2018, YOUR NAME
%%
%%          Notes:  
%%
%%=====================================================================================

\chapter{Introducción}
\lipsum[1-3]



%  Capítulo 2
%  -----------------------------------------------------------------------------

%%=====================================================================================
%%
%%       Filename:  cap2.tex
%%
%%    Description:  
%%
%%        Version:  1.0
%%        Created:  11/22/2018
%%       Revision:  none
%%
%%         Author:  YOUR NAME (), 
%%   Organization:  
%%      Copyright:  Copyright (c) 2018, YOUR NAME
%%
%%          Notes:  
%%
%%=====================================================================================

\section{Segundo apartado}
   \lipsum[1]
   \subsection{Esto es sorprendente}
   \lipsum[2]
\section{Tercer apartado}
\subsection{Esto es sorprendente}
\lipsum[3-4]
\subsubsection{Un subtítulo interesante}
\lipsum[5-7]


%  Capítulo 3
%  -----------------------------------------------------------------------------

%%=====================================================================================
%%
%%       Filename:  cap3.tex
%%
%%    Description:  
%%
%%        Version:  1.0
%%        Created:  11/22/2018
%%       Revision:  none
%%
%%         Author:  YOUR NAME (), 
%%   Organization:  
%%      Copyright:  Copyright (c) 2018, YOUR NAME
%%
%%          Notes:  
%%
%%=====================================================================================

\chapter{Otro título no esperado}
   \lipsum[1-3]
\section{Primer apartado}
   \lipsum[4-5]
\subsection{Un subtitulo interesante}
   \lipsum[6]
\subsubsection{Esto es sorprendente}
   \lipsum[2-3]
   En la imagen \ref{fig:proyecto_tres} se visualiza el logo del proyecto \LaTeX.

\begin{figure}[ht]
   \centering
   \includegraphics[width=6cm]{latex.png}
   \caption{Logo del proyecto \LaTeX}
   \label{fig:proyecto_tres}
\end{figure}
\section{Segundo apartado}
   \lipsum[1]
   \subsection{Esto es sorprendente}
   \lipsum[2]
\section{Tercer apartado}
\subsection{Esto es sorprendente}
\lipsum[3-4]
\subsubsection{Un subtítulo interesante}
\lipsum[5-7]


%  Capítulo 4
%  -----------------------------------------------------------------------------
%%=====================================================================================
%%
%%       Filename:  cap4.tex
%%
%%    Description:  
%%
%%        Version:  1.0
%%        Created:  11/22/2018
%%       Revision:  none
%%
%%         Author:  YOUR NAME (), 
%%   Organization:  
%%      Copyright:  Copyright (c) 2018, YOUR NAME
%%
%%          Notes:  
%%
%%=====================================================================================

\chapter{Un capitulo especial}
\section{Primer apartado}
   \lipsum[4-5]
\subsection{Un subtitulo interesante}
   \lipsum[6]
\subsubsection{Esto es sorprendente}
   \lipsum[2-3]
   En la imagen \ref{fig:proyecto_dos} se visualiza el logo del proyecto \LaTeX.

\begin{figure}[ht]
   \centering
   \includegraphics[width=6cm]{latex.png}
   \caption{Logo del proyecto \LaTeX}
   \label{fig:proyecto_dos}
\end{figure}
\section{Segundo apartado}
   \lipsum[1]
   \subsection{Esto es sorprendente}
   \lipsum[2]
\section{Tercer apartado}
\subsection{Esto es sorprendente}
\lipsum[3-4]
\subsubsection{Un subtítulo interesante}
\lipsum[5-7]
El contenido de este taller se encuentra en las presentaciones adjuntas que se
complementan con el material de apoyo presentes en el capítulo 2 de este
documento.\cite{Met78}

El contenido de este taller se encuentra en las presentaciones adjuntas que se
complementan con el material de apoyo presentes en el capítulo 2 de este
documento.\cite{Wel03}

El contenido de este taller se encuentra en las presentaciones adjuntas que se
complementan con el material de apoyo presentes en el capítulo 2 de este
documento.\cite{Lit96}

El contenido de este taller se encuentra en las presentaciones adjuntas que se
complementan con el material de apoyo presentes en el capítulo 2 de este
documento.\cite{Tho98w}





% ------------------------------------------------------------------------------
%  BIBLIOGRAFÍA
% ------------------------------------------------------------------------------

\bibliographystyle{apacite}
\bibliography{myrefs.bib}


\end{document}
